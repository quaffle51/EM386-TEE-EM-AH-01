% !TeX root = ./TMA01.tex
Case 1: If the last digit of the codeword is an erasure then the value of the erasure, $e_{10}$, is given by:
\[
	e_{10} = \sum_{i=1}^9 ix_i \Mod{11}.
\]
Otherwise the erasure, $e_j$, will have occurred at location $j$, $j \in \{ 0,1,\ldots,9 \}$ in which case the value of $e_j$ is given by \eqref{eq:1.2}:
\begin{align}
\label{eq:1.2}
	\sum_{i=1}^{j-1}ix_i  + je_j + \sum_{i=j+1}^{10}ix_i&\equiv 0 \Mod{11},\nonumber\\
	\left(\sum_{i=1}^{j-1}ix_i + \sum_{i=j+1}^{10}ix_i\right)&\equiv -je_j \Mod{11},\nonumber\\
	j^{-1}\left(\sum_{i=1}^{j-1}ix_i + \sum_{i=j+1}^{10}ix_i\right)&\equiv -e_j \Mod{11},\nonumber\\
	-j^{-1}\left(\sum_{i=1}^{j-1}ix_i + \sum_{i=j+1}^{10}ix_i\right) \Mod{11} &\equiv e_j,
\end{align}
where $j^{-1}$ is the multiplicative inverse of j in $\Z_{11}$. So ISBN-10 is 1-erasure correcting.

Case 2: Assume two erasures have occurred, $e_j$ and $e_k$, at locations $j$ and $k$ ($j\neq k, k>j$) in the codeword. Then, we have:
\begin{align}
\label{eq:1.3}
	\sum_{i=1}^{j-1}ix_i + je_j + \sum_{i=j+1}^{k-1}ix_i + ke_k + \sum_{i=k+1}^{10}ix_i \equiv 0 \Mod{11}.
\end{align}
The congruence modulo 11~\eqref{eq:1.3} has two unknowns and therefore it is not possible to uniquely determine both erasures. Thus, ISBN-10 is not 2-erasure correcting.

Given the received vector $04862?263X$, where ? denotes an undecodeable symbol, the correct ISBN-10 can be determined using \eqref{eq:1.2} as follows.

The erasure has occurred at location six in the received vector therefore $j=6$ in this case.
\[
	-\left[6^{-1}\left(\sum_{i=1}^{5}ix_i + \sum_{i=7}^{10}ix_i \right)\right] \Mod{11} \equiv e_6.
\]
The multiplicative inverse of 6 in GF(11) is 2, so we have\marginnote{\textbf{H}~p.36}[-0.5cm]%
\[
	-2\left(\sum_{i=1}^{5}ix_i + \sum_{i=7}^{10}ix_i \right) \Mod{11} \equiv e_6.
\]
Now,
\[
	\sum_{i=1}^{5}ix_i = 0 + 2\cdot4 +3\cdot8 +4\cdot6 + 5\cdot2 = \pyc{print(0+2*4+3*8+4*6+5*2)},
\]
and
\[
	\sum_{i=7}^{10}ix_i = 7\cdot2 +8\cdot6 +9\cdot3 + 10\cdot10 = \pyc{print(7*2+8*6+9*3+10*10)}.
\]
Therefore,
\[
  -2(\pyc{print(0+2*4+3*8+4*6+5*2)} + \pyc{print(7*2+8*6+9*3+10*10)}) \Mod{11}\equiv e_6,
\]
\[
	-2(\pyc{print(0+2*4+3*8+4*6+5*2+7*2+8*6+9*3+10*10)}) \Mod{11} \equiv e_6,
\]
\[
	-\pyc{print(2*(0+2*4+3*8+4*6+5*2+7*2+8*6+9*3+10*10))} \Mod{11} \equiv e_6,
\]
\[
	\pyc{print(-2*(0+2*4+3*8+4*6+5*2+7*2+8*6+9*3+10*10)%11)} = e_6.
\]
So, $e_6 = \pyc{print(11-2*(0+2*4+3*8+4*6+5*2+7*2+8*6+9*3+10*10)%11)}$.

Check:
\pyc{print(isbn_like_6 + r'\endinput')}

\pyc{print(valid_6 + r'\endinput')}
