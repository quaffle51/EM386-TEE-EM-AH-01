% !TeX root = ./TMA01.tex
\begin{pycode}
a = math.comb(10, 1)*11**7
b = math.comb(10, 2)*11**6
c = math.comb(10, 3)*11**5
d = math.comb(10, 4)*11**4
e = math.comb(10, 5)*11**3
f = math.comb(10, 6)*11**2
g = math.comb(10, 7)*11**1
h = math.comb(10, 8)*11**0

y = a -b + c- d +e -f +g -h
z = 11**8 - y
\end{pycode}
Assume that there can be ten sixes in a codeword of $C$ in which case both parity check equation should be zero and the codeword is valid. It can be easily shown that
\begin{equation}
\label{eq:4.10}
	A = \sum_{i=1}^{10} i6\equiv \pyc{print(1*6+2*6+3*6+4*6+5*6+6*6+7*6+8*6+9*6+10*6)} \equiv \pyc{Mod(1*6+2*6+3*6+4*6+5*6+6*6+7*6+8*6+9*6+10*6,11)}\Mod{11},
\end{equation}
and
\begin{equation}
\label{eq:4.10}
	B = \sum_{i=1}^{10} 6 \equiv 60 \equiv \pyc{Mod(60,11)} \Mod{11}.
\end{equation}
So according to \hill page~78 if $A=0$ or $B=0$ but not both, then at least two errors have been detected.  If two errors have occurred then that means there cannot be nine or ten sixes in any codeword of $C$. Note that the transposition of two digits does not affect the outcome and therefore can be discounted.

Using a similar argument to that of part~(d) above but this time for the $[10,8]$-code defined over $GF\{11\}$ we can summarise the previous argument in Table~\ref{tab:17} proving that there are $\pyc{print(z)}$ codewords in $C$.

Thus, from the summary Table~\ref{tab:17} it is proved that there are $82644629$ codewords in $C$ as stated by \hill page~76.
\begin{table}[!htp]\centering
\begin{tabular}{lrrrrrr}\toprule
$n$ &$k$ &$a=$ $n\choose k$ &$b=11^{8-k}$ &sign &$\textrm{sign}\times a\times b$ \\\midrule
10 &1 &10 &19487171 &1 &194871710 \\
10 &2 &45 &1771561 &-1 &-79720245 \\
10 &3 &120 &161051 &1 &19326120 \\
10 &4 &210 &14641 &-1 &-3074610 \\
10 &5 &252 &1331 &1 &335412 \\
10 &6 &210 &121 &-1 &-25410 \\
10 &7 &120 &11 &1 &1320 \\
10 &8 &45 &1 &-1 &-45 \\
\multicolumn{5}{c}{} &131714252 \\
\multicolumn{6}{r}{$11^7 - 131714252 = 82644629$} \\
\bottomrule
\end{tabular}
\caption{Summary of derivation of the number of codewords in $C$ using a similiar derivation to that in part~(d).}\label{tab:17}
\rule{\textwidth}{2pt}
\end{table}

