% !TeX root = ./TMA01.tex
To determine which of the following codes are linear over the alphabet
indicated use will be made of conditions (1) and (2) of \hill on page 47.
\qsubpart
\[
	C_1 = \{00000, 11001, 10011, 01010\} \textrm{\ over\ } \Z_2
\]
To check this, we have to show that $C_1$ is closed under addition and scalar multiplication, so we have for scalar multiplication
\[
	\textbf{0}c = 00000,\qquad 1c = c
\]
for any codeword $c$ in $C_1$. For addition we have
\[
	c + c = 00000 \textrm{\ and\ } c + 00000 = c
\]
for any codeword $c$ in $C_1$ and
\begin{align*}
	11001 + 10011 = 01010 \in C_1,\\
	11001 + 01010 = 10011 \in C_1,\\
	10011 + 01010 = 11001 \in C_1.
\end{align*}
In view of the forgoing $C_1$ is linear as it passes both conditions of \hill on page 47.
\qsubpart
\[
	C_2 = \{000, 001, 010, 100\} \textrm{\ over\ } \Z_2
\]
\begin{align*}
	 001 + 010 = 011 \not\in C_2.
\end{align*}
$C_2$ is not linear as it fails under condition (1) of \hill on page 47.
\qsubpart
\[
	C_3 = {00000, 11001, 10011, 01010} \textrm{\ over\ } \Z_3
\]
\begin{align*}
	11001 + 10011 = 21012 \not\in C_3.
\end{align*}
$C_3$ is not linear as it fails under condition (1) of \hill on page 47..