% !TeX root = ./TMA01.tex
The code $D$ is a linear $[6,4]$-code over $GF(7)=\{0,1,\ldots,6\}$ and the codewords $\bm{x}=x_1x_2x_3x_4x_5x_6$ are defined by two parity-check equations
\[
	\sum_{i=1}^6 x_i\equiv 0 \Mod{7}\quad\textrm{and}\quad \sum_{i=1}^6 ix_i\equiv 0 \Mod{7}
\]
The algorithm that will correct any single transmission error in a codeword of $D$ is developed as follows.

Assume that a single transmission error has occurred in the $j^{th}$ position of the received codeword. Then the value of $j \in \{1,2,\ldots,6\}$ can be found as follows.

For the codeword as transmitted:
\begin{align}
\label{eq:4.1}
	\sum_{i=1,i\neq j}^{6} x_i + x_j \equiv 0 \Mod{7},
\end{align}
\begin{align}
\label{eq:4.2}
	\sum_{i=1,i\neq j}^{6} ix_i + jx_j \equiv 0 \Mod{7}.
\end{align}
For the codeword as received assuming a single transmission error has occurred at coordinate position $j$ and that $x_j$ has been changed to $y_j$, then we have:
\begin{align}
\label{eq:4.3}
	\sum_{i=1,i\neq j}^{6} x_i + y_j \equiv a \Mod{7},
\end{align}
\begin{align}
\label{eq:4.4}
	\sum_{i=1,i\neq j}^{6} ix_i + jy_j \equiv b \Mod{7}.
\end{align}
$\eqref{eq:4.3} - \eqref{eq:4.1}$ gives \eqref{eq:4.5}
\begin{align}
\label{eq:4.5}
	y_j - x_j \equiv a \Mod{7}.
\end{align}
$\eqref{eq:4.4} - \eqref{eq:4.2}$ gives \eqref{eq:4.6}
\begin{align}
\label{eq:4.6}
	jy_j - jx_j  \equiv b \Mod{7}.
\end{align}
$\eqref{eq:4.5}\times j$ gives \eqref{eq:4.7}
\begin{align}
\label{eq:4.7}
	jy_j - jx_j \equiv ja \Mod{7}.
\end{align}
$\eqref{eq:4.7} - \eqref{eq:4.6}$ gives \eqref{eq:4.8}
\begin{align}
\label{eq:4.8}
	ja - b \equiv 0 \Mod{7},\nonumber\\
	ja \equiv b \Mod{7},\nonumber\\
	j \equiv a^{-1}b \Mod{7},
\end{align}
where $a^{-1}$ is the multiplicative inverse of $a$ in $GF\{7\}$.
\begin{table}[!htp]\centering
\begin{tabular}{|l|rrrrrrr|r}\toprule
$\times$ &0 &1 &2 &3 &4 &5 &6 \\\midrule
0 &0 &0 &0 &0 &0 &0 &0 \\
1 &0 &1 &2 &3 &4 &5 &6 \\
2 &0 &2 &4 &6 &1 &3 &5 \\
3 &0 &3 &6 &2 &5 &1 &4 \\
4 &0 &4 &1 &5 &2 &6 &3 \\
5 &0 &5 &3 &1 &6 &4 &2 \\
6 &0 &6 &5 &4 &3 &2 &1 \\
\bottomrule
\end{tabular}
\caption{$GF\{7\}$ multiplication table to find the multiplicative inverse of $j$.}\label{tab:16}
\end{table}
From \eqref{eq:4.2} knowing $j$ then $x_j$ is found as follows
\begin{align}
\label{eq:4.9}
	\sum_{i=1,i\neq j}^{6} ix_i + jx_j \equiv 0 \Mod{7},\nonumber\\
	\sum_{i=1,i\neq j}^{6} ix_i \equiv -jx_j \Mod{7}, \nonumber\\
	-j^{-1}\sum_{i=1,i\neq j}^{6} ix_i \equiv x_j \Mod{7}.
\end{align}
Now assume that two digits of the received codeword have been transposed. Then regardless of which two digits were transposed $\sum_{i=1}^6 x_i\equiv 0 \Mod{7}$ is still true. 

Let the two digits that were transposed be the ones at the $j^{th}$ and $k^{th}$ coordinate positions in the received codeword where $k>j$ and $j,k \in \{1,2,\ldots,6\}$ and further suppose that $x_i \neq x_j$. Then,
\begin{align*}
	\sum_{i=1,i\neq j,i\neq k}^{j-1}ix_i + kx_j + jx_k \not\equiv 0 \Mod{7}.
\end{align*}
Thus, algorithm is as follows.
\begin{itemize}
	\item If \eqref{eq:4.1} and \eqref{eq:4.2} are true then no errors have occurred and we therefore accept the received codeword.
	\item If \eqref{eq:4.1} is true but not \eqref{eq:4.2} then a transposition error has occurred so request retransmission of the codeword as it cannot be corrected.
	\item If both \eqref{eq:4.1} and \eqref{eq:4.2} are not true then assume a single error. Use \eqref{eq:4.8} to determine the location, $j$, of the error in the codeword and then use \eqref{eq:4.9} in conjunction with Table~\ref{tab:16} to find the multiplicative inverse of $j$ to determine the correct digit.
\end{itemize}
\qsubpart
Given the received vector is $\bm{x} = 113235$ then using \eqref{eq:4.1} we have
\begin{pycode}
import math
codeword = "113235"
sum = 0
sum1 = 0
sum2 = 0
for i in range(len(codeword)):
	sum += int(codeword[i])
	sum1 += (i+1)*int(codeword[i])
	if (i+1) != 2:
		sum2 += (i+1)*int(codeword[i])
\end{pycode}
\begin{align*}
	a = \sum_{i=1}^{6} x_i \equiv \pyc{print(sum)} \Mod{7} \equiv \pyc{print(int(math.fmod(sum,7)))}  \Mod{7},
\end{align*}
and
\begin{align*}
	b = \sum_{i=1}^{6} ix_i \equiv \pyc{print(sum1)} \Mod{7} \equiv \pyc{print(int(math.fmod(sum1,7)))} \Mod{7}.
\end{align*}
Using \eqref{eq:4.8}
\begin{align*}
	j = a^{-1} b \Mod{7} = 1^{-1}\times 2 \Mod{7} \equiv 2 \Mod{7}.
\end{align*}
So the digit at location $2$ is the one that is in error. To determine the correct digit use \eqref{eq:4.9}.
\begin{align*}
	x_2 = -\sum_{i=1,i\neq 2}^{6} ix_i \equiv \pyc{print(-sum2)} \Mod{7} \equiv \pyc{print(int(math.fmod(-sum2,7)))}.
\end{align*}
Thus, the correct codeword is $103235$.
\begin{pycode}
codeword = "625152"
sum  = 0
sum1 = 0
for i in range(len(codeword)):
	sum += int(codeword[i])
	sum1 += (i+1)*int(codeword[i])
\end{pycode}
\qsubpart
Given the received codeword is \pyc{print(codeword)} then in this case 
$\pyc{print("a={}".format(int(math.fmod(sum,7))))}$ and 
$\pyc{print("b={}".format(int(math.fmod(sum1,7))))}$ so by the above algorithm a transposition error has occurred and so a request for retransmission of the codeword should be sent.

