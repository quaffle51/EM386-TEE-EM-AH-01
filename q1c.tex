% !TeX root = ./TMA01.tex
If the ISBN-10 code is modified by appending to each codeword $\textbf{x}=x_1x_2\cdots x_{10}$ an eleventh digit $x_{11}$ given by $x_{11}=\sum_{i=1}^{10} x_i \Mod{11}$ then the minimum distance of the resulting code will be three. This can be justified as follows.

Assume that the digit in the $j^{th}$  $(1\leq j \leq 9)$ position of the codeword is changed from $x_j$ to $y_j$ where $x_j,y_j \in \Z_{10}$ and $x_j \not= y_j$, then
\begin{align}
\label{eq:1.4}
	\left.x_{10}\right._{(j,x_{j})} &\equiv \sum_{\substack{i=1\\ i\neq j}}^9 ix_i + jx_j \Mod{11},\nonumber\\
	\left.x_{10}\right._{(j,y_{j})} &\equiv \sum_{\substack{i=1\\ i\neq j}}^9 ix_i + jy_j \Mod{11},\nonumber\\
	\left.x_{10}\right._{(j,y_{j})} - \left.x_{10}\right._{(j,x_{j})} &\equiv j(y_j-x_j) \Mod{11}.
\end{align}
\eqref{eq:1.4} Shows that the digit in the tenth position of the codeword $x_{10}$ will change when any one of the digits $x_i, i\in\{1,2,\ldots,9\}$ changes.

Now, to see what happens to the eleventh digit when one of the digits $x_i, i\in\{1,2,\ldots,9\}$ changes.
\begin{align}
\label{eq:1.5}
	\left.x_{11}\right._{(j,x_j)} &\equiv \sum_{i=1, i\neq j}^9 x_i + x_j + \left.x_{10}\right._{(j,x_j)}\Mod{11},\nonumber\\
	\left.x_{11}\right._{(j,y_j)} &\equiv \sum_{i=1, i\neq j}^9 x_i + y_j + \left.x_{10}\right._{(j,y_j)}\Mod{11},\nonumber\\
	\left.x_{11}\right._{(j,y_j)}-\left.x_{11}\right._{(j,x_j)} &\equiv (y_j-x_j) +\left.x_{10}\right._{(j,y_j)}-\left.x_{10}\right._{(j,x_j)}\Mod{11},\nonumber\\
	\left.x_{11}\right._{(j,y_j)}-\left.x_{11}\right._{(j,x_j)} &\equiv (y_j-x_j) + j(y_j-x_j) \Mod{11},\nonumber\\
	\left.x_{11}\right._{(j,y_j)}-\left.x_{11}\right._{(j,x_j)} &\equiv (y_j-x_j)(1 + j) \Mod{11}.
\end{align}
\eqref{eq:1.5} shows that the digit in the eleventh position of the codeword $x_{11}$ will change when any one of the digits $x_i, i\in\{1,2,\ldots,9\}$ changes. 

Thus, if one of the digits $x_i, i\in\{1,2,\ldots,9\}$ changes, both $x_{10}$ and $x_{11}$ change showing that the minimum distance of this modified ISBN-10 is three.

\qsubpart
To find $x_4$:
\begin{align}
\label{eq:1.6}
	\sum_{i=1}^{3}ix_i + 4x_4+\sum_{i=5}^{10} ix_i &\equiv 0 \Mod{11}\nonumber,\\
	\sum_{i=1}^{3}x_i + x_4+\sum_{i=5}^{10} x_i &\equiv 9 \Mod{11}\nonumber,\\
	\sum_{i=1}^{3}x_i(i-1) + 3x_4+\sum_{i=5}^{10} x_i(i-1) &\equiv -9 \Mod{11}\nonumber,\\
	\sum_{i=1}^{3}x_i(i-1) +\sum_{i=5}^{10} x_i(i-1) +9 &\equiv -3x_4 \Mod{11}\nonumber,\\
	-3^{-1}\left(\sum_{i=1}^{3}x_i(i-1)+\sum_{i=5}^{10} x_i(i-1) +9\right) &\equiv x_4 \Mod{11}\nonumber,\\
	-4\left(\sum_{i=1}^{3}x_i(i-1) +\sum_{i=5}^{10} x_i(i-1) +9\right) &\equiv x_4 \Mod{11}.
\end{align}
Given the received vector $\pyc{print(isbn_i)}$ and applying \eqref{eq:1.6} to find $x_4$:
\begin{align*}
	-4(23 + 183 + 9) \equiv x_4 \Mod{11}\\
	\pyc{print(-4*(23 + 183 + 9))}\equiv x_4 \Mod{11}\\
	9\equiv x_4 \Mod{11}\\
\end{align*}
Therefore, the corrected vector is $297\textbf{9}2357099$.
\qsubpart
To find $x_5$:
\begin{align}
\label{eq:1.7}
	\sum_{\substack{i=1, i\neq5, i\neq7}}^{10}ix_i +5x_5 + 7x_7 &\equiv 0 \Mod{11},\nonumber\\
	\sum_{\substack{i=1, i\neq5, i\neq7}}^{10}x_i  + x_5 +  x_7 &\equiv x_{11} \Mod{11},\nonumber\\
	\sum_{\substack{i=1, i\neq5, i\neq7}}^{10}7x_i  + 7x_5 + 7 x_7 &\equiv 7x_{11} \Mod{11},\nonumber\\
	\sum_{\substack{i=1, i\neq5, i\neq7}}^{10}(7-i)x_i + 2x_5 &\equiv 7x_{11} \Mod{11},\nonumber\\
	\sum_{\substack{i=1, i\neq5, i\neq7}}^{10}(7-i)x_i- 7x_{11} &\equiv -2x_5  \Mod{11},\nonumber\\
	-6\left(\sum_{\substack{i=1, i\neq5, i\neq7}}^{10}(7-i)x_i- 7x_{11}\right) &\equiv x_5  \Mod{11},\nonumber\\
\end{align}
Similarly, to find $x_7$:
\begin{align}
\label{eq:1.8}
	\sum_{\substack{i=1, i\neq5, i\neq7}}^{10}ix_i +5x_5 + 7x_7 &\equiv 0 \Mod{11},\nonumber\\
	\sum_{\substack{i=1, i\neq5, i\neq7}}^{10}x_i  + x_5 +  x_7 &\equiv x_{11} \Mod{11},\nonumber\\
	\sum_{\substack{i=1, i\neq5, i\neq7}}^{10}5x_i  + 5x_5 + 5 x_7 &\equiv 5x_{11} \Mod{11},\nonumber\\
	\sum_{\substack{i=1, i\neq5, i\neq7}}^{10}(5-i)x_i - 2x_7 &\equiv 5x_{11} \Mod{11},\nonumber\\
	\sum_{\substack{i=1, i\neq5, i\neq7}}^{10}(5-i)x_i - 5x_{11} &\equiv 2x_{7} \Mod{11},\nonumber\\
	6\left(\sum_{\substack{i=1, i\neq5, i\neq7}}^{10}(5-i)x_i - 5x_{11}\right) &\equiv x_{7} \Mod{11}.
\end{align}
Applying \eqref{eq:1.7} and \eqref{eq:1.8} to the received vector gives
\begin{align*}
	x_5 &\equiv -6(64+7-6 -7(1)) \equiv -6(58) \equiv -348 \equiv -7 \equiv 4 \Mod{11},\textrm{ and }\\
	x_7 &\equiv 6(30-7 -14 -5(1)) \equiv 6(4) \equiv 24 \equiv 2 \Mod{11}.
\end{align*}
So, $x_5=4$ and $x_7=2$.  Thus, the corrected received vector is $2159\textbf{4}7\textbf{2}3011$.
\qsubpart
From part~(ii) we solved for two unknowns using the two equations \eqref{eq:1.7} and \eqref{eq:1.8} and therefore unique values were found for the two unknowns.  In view of this, it is not possible to solve uniquely for three unknowns having access only to these two equations.  Three equations would be needed to solve uniquely for three unknowns. 
