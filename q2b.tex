% !TeX rt = ./TMA01.tex
Using Theorem 5.4 of \hill page 50 $G_2$ can be obtained from $G_1$ in the following way:
\begin{alignat*}{2}
\begin{sysmatrix}{rrrr}
 0 &  0 & 1 & 1 \\
 0 &  1 & 1 & 0 \\
 1 &  1 & 0 & 0
\end{sysmatrix}
&\!\begin{aligned}
\ro{r_2+r_1}\\
\end{aligned}
\begin{sysmatrix}{rrrr}
 0 &  0 & 1 & 1 \\
 0 &  1 & 0 & 1 \\
 1 &  1 & 0 & 0
\end{sysmatrix}
&\ro{r_3 + r_2}
\begin{sysmatrix}{rrrr}
 0 &  0 & 1 & 1 \\
 0 &  1 & 0 & 1 \\
 1 &  0 & 0 & 1
\end{sysmatrix}
\\
&\!\begin{aligned}
\ro{\textrm{swap $r_1$ and $r_3$\ }}
\end{aligned}
\begin{sysmatrix}{rrrr}
 1 &  0 & 0 & 1\\
 0 &  1 & 0 & 1 \\
 0 &  0 & 1 & 1
\end{sysmatrix}
\end{alignat*}
Thus, $G_1$ and $G_2$ are equivalent as linear codes.