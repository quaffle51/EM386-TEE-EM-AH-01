% !TeX root = ./TMA01.tex
The Table~\ref{tab:15} is used for incomplete decoding and shows a top part and a bottom part.
\begin{table}[!htp]\centering
\begin{tabular}{lrrrrrrrr}\toprule
000000 &100110 &010101 &001111 &110011 &101001 &011010 &111100 \\
100000 &000110 &110101 &101111 &010011 &001001 &111010 &011100 \\
010000 &110110 &000101 &011111 &100011 &111001 &001010 &101100 \\
001000 &101110 &011101 &000111 &111011 &100001 &010010 &110100 \\
000100 &100010 &010001 &001011 &110111 &101101 &011110 &111000 \\
000010 &100100 &010111 &001101 &110001 &101011 &011000 &111110 \\
000001 &100111 &010100 &001110 &110010 &101000 &011011 &111101 \\
\midrule
110000 &010110 &100101 &111111 &000011 &011001 &101010 &001100 \\
\bottomrule
\end{tabular}
\caption{A partitioned Slepian standard array for $C$}\label{tab:15}
\end{table}
$d(C) = 2t + 1 =3$ for $C$ in this case and $t=1$. So, the incomplete decoding scheme guarantees the correction of $\leq t, \leq 1$ errors in any codeword.
\qsubpart
Given the received vector $\bm{y} = 111111$ which appears in the bottom part of Table~\ref{tab:15} we conclude that more than one error has occurred during transmission and request the resending of the codeword.
\qsubpart
Given the received vector $\bm{y} =110011$ which appears in the top row of the upper part of Table~\ref{tab:15} we conclude that no errors have occurred and decode the received vector as the codeword $110011$.
\qsubpart
Given the received vector $\bm{y} = 111101$ which appears in the upper part of Table~\ref{tab:15}, in a row other than the top row, we conclude that one error has occurred and decode the vector as the codeword $111100$.